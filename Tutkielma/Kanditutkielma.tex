% TIETOTEKNIIKAN KANDIDAATINTUTKIELMA
\documentclass[utf8,bachelor]{gradu3}
\usepackage[bookmarksopen,bookmarksnumbered,linktocpage]{hyperref}
\addbibresource{Lahteet.bib}

% META

\begin{document}

\title{Reaaliaikaisesti renderöidyn vektorigrafiikan käyttö videopeleissä}
\translatedtitle{Realtime rendered vector graphics in videogames}

\avainsanat{avain1, avain2, avain3}
\keywords{avainsanat englanniksi}
\tiivistelma{
    Tiivistelmä on tyypillisesti 5-10 riviä pitkä esitys työn pääkohdista (tausta, tavoite, tulokset, johtopäätökset).
}
\abstract{
    Englanninkielinen versio tiivistelmästä.
}

\author{Ville Kankaanpää}
\contactinformation{\texttt{vivakank@student.jyu.fi}}
%\supervisor{}

% Otsikko ja metatiedot
\maketitle

% Sisällysluettelo
\mainmatter

% Teksti

\chapter{Johdanto}

\chapter{Vektorigrafiikka}

\section{Reaaliaikainen vektorigrafiikan renderöinti}

\section{Vektorigrafiikan ja rasterigrafiikan eduista ja haitoista}

\chapter{Vektorigrafiikat videopeleissä}

\section{Vektorigrafiikan historia peleissä}

\section{Vektorigrafiikan nykytila peleissä}

\section{Ongelmia vektorigrafiikan käytössä}

\chapter{Pohdintaa vektorigrafiikoiden tulevaisuudesta}

\section{Selainpelit}

\section{Vektorigrafiikka 3D-kappaleiden tekstuureissa}

\chapter{Yhteenveto}

\printbibliography

\end{document}